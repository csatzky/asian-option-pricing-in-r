% Options for packages loaded elsewhere
\PassOptionsToPackage{unicode}{hyperref}
\PassOptionsToPackage{hyphens}{url}
\PassOptionsToPackage{dvipsnames,svgnames*,x11names*}{xcolor}
%
\documentclass[
  12pt,
]{article}
\usepackage[sc, osf]{mathpazo}
\usepackage{amssymb,amsmath}
\usepackage{ifxetex,ifluatex}
\ifnum 0\ifxetex 1\fi\ifluatex 1\fi=0 % if pdftex
  \usepackage[T1]{fontenc}
  \usepackage[utf8]{inputenc}
  \usepackage{textcomp} % provide euro and other symbols
\else % if luatex or xetex
  \usepackage{unicode-math}
  \defaultfontfeatures{Scale=MatchLowercase}
  \defaultfontfeatures[\rmfamily]{Ligatures=TeX,Scale=1}
\fi
% Use upquote if available, for straight quotes in verbatim environments
\IfFileExists{upquote.sty}{\usepackage{upquote}}{}
\IfFileExists{microtype.sty}{% use microtype if available
  \usepackage[]{microtype}
  \UseMicrotypeSet[protrusion]{basicmath} % disable protrusion for tt fonts
}{}
\makeatletter
\@ifundefined{KOMAClassName}{% if non-KOMA class
  \IfFileExists{parskip.sty}{%
    \usepackage{parskip}
  }{% else
    \setlength{\parindent}{0pt}
    \setlength{\parskip}{6pt plus 2pt minus 1pt}}
}{% if KOMA class
  \KOMAoptions{parskip=half}}
\makeatother
\usepackage{xcolor}
\IfFileExists{xurl.sty}{\usepackage{xurl}}{} % add URL line breaks if available
\IfFileExists{bookmark.sty}{\usepackage{bookmark}}{\usepackage{hyperref}}
\hypersetup{
  colorlinks=true,
  linkcolor=blue,
  filecolor=Maroon,
  citecolor=Blue,
  urlcolor=blue,
  pdfcreator={LaTeX via pandoc}}
\urlstyle{same} % disable monospaced font for URLs
\usepackage[margin=1in]{geometry}
\usepackage{color}
\usepackage{fancyvrb}
\newcommand{\VerbBar}{|}
\newcommand{\VERB}{\Verb[commandchars=\\\{\}]}
\DefineVerbatimEnvironment{Highlighting}{Verbatim}{commandchars=\\\{\}}
% Add ',fontsize=\small' for more characters per line
\usepackage{framed}
\definecolor{shadecolor}{RGB}{248,248,248}
\newenvironment{Shaded}{\begin{snugshade}}{\end{snugshade}}
\newcommand{\AlertTok}[1]{\textcolor[rgb]{0.94,0.16,0.16}{#1}}
\newcommand{\AnnotationTok}[1]{\textcolor[rgb]{0.56,0.35,0.01}{\textbf{\textit{#1}}}}
\newcommand{\AttributeTok}[1]{\textcolor[rgb]{0.77,0.63,0.00}{#1}}
\newcommand{\BaseNTok}[1]{\textcolor[rgb]{0.00,0.00,0.81}{#1}}
\newcommand{\BuiltInTok}[1]{#1}
\newcommand{\CharTok}[1]{\textcolor[rgb]{0.31,0.60,0.02}{#1}}
\newcommand{\CommentTok}[1]{\textcolor[rgb]{0.56,0.35,0.01}{\textit{#1}}}
\newcommand{\CommentVarTok}[1]{\textcolor[rgb]{0.56,0.35,0.01}{\textbf{\textit{#1}}}}
\newcommand{\ConstantTok}[1]{\textcolor[rgb]{0.00,0.00,0.00}{#1}}
\newcommand{\ControlFlowTok}[1]{\textcolor[rgb]{0.13,0.29,0.53}{\textbf{#1}}}
\newcommand{\DataTypeTok}[1]{\textcolor[rgb]{0.13,0.29,0.53}{#1}}
\newcommand{\DecValTok}[1]{\textcolor[rgb]{0.00,0.00,0.81}{#1}}
\newcommand{\DocumentationTok}[1]{\textcolor[rgb]{0.56,0.35,0.01}{\textbf{\textit{#1}}}}
\newcommand{\ErrorTok}[1]{\textcolor[rgb]{0.64,0.00,0.00}{\textbf{#1}}}
\newcommand{\ExtensionTok}[1]{#1}
\newcommand{\FloatTok}[1]{\textcolor[rgb]{0.00,0.00,0.81}{#1}}
\newcommand{\FunctionTok}[1]{\textcolor[rgb]{0.00,0.00,0.00}{#1}}
\newcommand{\ImportTok}[1]{#1}
\newcommand{\InformationTok}[1]{\textcolor[rgb]{0.56,0.35,0.01}{\textbf{\textit{#1}}}}
\newcommand{\KeywordTok}[1]{\textcolor[rgb]{0.13,0.29,0.53}{\textbf{#1}}}
\newcommand{\NormalTok}[1]{#1}
\newcommand{\OperatorTok}[1]{\textcolor[rgb]{0.81,0.36,0.00}{\textbf{#1}}}
\newcommand{\OtherTok}[1]{\textcolor[rgb]{0.56,0.35,0.01}{#1}}
\newcommand{\PreprocessorTok}[1]{\textcolor[rgb]{0.56,0.35,0.01}{\textit{#1}}}
\newcommand{\RegionMarkerTok}[1]{#1}
\newcommand{\SpecialCharTok}[1]{\textcolor[rgb]{0.00,0.00,0.00}{#1}}
\newcommand{\SpecialStringTok}[1]{\textcolor[rgb]{0.31,0.60,0.02}{#1}}
\newcommand{\StringTok}[1]{\textcolor[rgb]{0.31,0.60,0.02}{#1}}
\newcommand{\VariableTok}[1]{\textcolor[rgb]{0.00,0.00,0.00}{#1}}
\newcommand{\VerbatimStringTok}[1]{\textcolor[rgb]{0.31,0.60,0.02}{#1}}
\newcommand{\WarningTok}[1]{\textcolor[rgb]{0.56,0.35,0.01}{\textbf{\textit{#1}}}}
\usepackage{graphicx,grffile}
\makeatletter
\def\maxwidth{\ifdim\Gin@nat@width>\linewidth\linewidth\else\Gin@nat@width\fi}
\def\maxheight{\ifdim\Gin@nat@height>\textheight\textheight\else\Gin@nat@height\fi}
\makeatother
% Scale images if necessary, so that they will not overflow the page
% margins by default, and it is still possible to overwrite the defaults
% using explicit options in \includegraphics[width, height, ...]{}
\setkeys{Gin}{width=\maxwidth,height=\maxheight,keepaspectratio}
% Set default figure placement to htbp
\makeatletter
\def\fps@figure{htbp}
\makeatother
\setlength{\emergencystretch}{3em} % prevent overfull lines
\providecommand{\tightlist}{%
  \setlength{\itemsep}{0pt}\setlength{\parskip}{0pt}}
\setcounter{secnumdepth}{5}
\renewcommand{\baselinestretch}{1.25}
\newcommand{\bquote}{\begin{quote}}
\newcommand{\equte}{\end{quote}}
\newcommand{\cc}{\centering}
\newcommand{\bdquote}{\begin{displayquote}}
\newcommand{\edquote}{\end{displayquote}}
\newcommand{\txtq}{\end{textquote}}
\usepackage{csquotes}
\usepackage{textcomp}
\usepackage{fontawesome5}
\usepackage{floatrow}
\floatsetup[figure]{capposition=top}
\floatsetup[table]{capposition=top}
\usepackage{setspace}
\newcommand{\inlinecode}{\texttt}
\newcommand{\typelatex}{\LaTeX}
\newcommand{\vspaceone}{\vspace{1 mm}}
\usepackage{booktabs}
\usepackage{longtable}
\usepackage{array}
\usepackage{multirow}
\usepackage{wrapfig}
\usepackage{float}
\usepackage{colortbl}
\usepackage{pdflscape}
\usepackage{tabu}
\usepackage{threeparttable}
\usepackage{threeparttablex}
\usepackage[normalem]{ulem}
\usepackage{makecell}
\usepackage{xcolor}

\title{Asian and European Option Specifics--\\
Pricing, Estimation, and Comparison Using R}
\author{By \href{https://www.linkedin.com/in/christian-satzky/}{Christian
Satzky, FRM}\\
Email: \href{mailto:c.satzky@gmail.com}{\nolinkurl{c.satzky@gmail.com}}}
\date{Last update: 20 October 2021}

\begin{document}
\maketitle

\newpage
\hypersetup{linkcolor=black}
\tableofcontents
\newpage
\hypersetup{linkcolor=blue}

\hypertarget{abstract}{%
\section*{Abstract}\label{abstract}}
\addcontentsline{toc}{section}{Abstract}

In this article, I present methods to efficiently estimate the price and
the probability of exercise for vanilla and exotic options. In addition,
I am using these results to compare the empirical delta between European
and average rate Asian options.

Results confirm that average rate exotic options generally reduce the
variability of the option's price, probability of exercise, and the
option's delta when compared to vanilla options. Hence, offering average
rate options might be beneficial when delta hedging the option portfolio
is of concern.

This research is 100\% reproducible. All \texttt{R} source code is
publicly available on my personal
\href{https://github.com/csatzky}{GitHub page}.\footnote{URL:
  \url{https://github.com/csatzky}}

\emph{Keywords: Exotic Option Pricing, Monte Carlo Simulation, Delta
Hedging.}

\newpage

\hypertarget{simulating-vanilla-options}{%
\section{Simulating Vanilla Options}\label{simulating-vanilla-options}}

Obtaining an option's price using Monte-Carlo simulation is
straightforward. However, simulating the delta of an option is more
complex. In the following, I am first reviewing the underlying price
process of the Black-Scholes model. Secondly, I present a simple,
lightweight \texttt{R} function to simulate this price process. Thirdly,
after carefully dissecting the Black-Scholes equation, I am presenting a
simulation approach on how to estimate an option's delta.

\hypertarget{geometric-brownian-motion}{%
\subsection{Geometric Brownian Motion}\label{geometric-brownian-motion}}

Under the Black-Scholes differential equation, the stock price \(S\)
follows a geometric Brownian motion (GBM). The stock price at time \(t\)
can be expressed as:

\begin{align}
S_t = S_0 \: \text{exp}\bigg(\bigg({\mu - \frac{\sigma^2}{2}\bigg)T + \sigma B_t} \bigg)
\end{align}

\begin{footnotesize}
Where:

\begin{tabular}{ll}

$S_t$ & Stock price at time t\\
$\mu$ & Annual rate of return of the risky asset $S$\\
$\sigma$ & Standard deviation of $\mu$\\
$B_t$ & Wiener process, $B_t \sim N(0,t), \forall t$\\

\end{tabular}
\end{footnotesize}

\hypertarget{simulating-a-geometric-brownian-motion}{%
\subsection{Simulating A Geometric Brownian
Motion}\label{simulating-a-geometric-brownian-motion}}

Let's denote \(\Delta t\) the fixed step size
\(\Delta t = t_i - t_{i-1} \: \forall i\), such that:

\begin{align}
S_{t_i} = S_{t_{i-1}} \: \text{exp}\bigg(\bigg({\mu - \frac{\sigma^2}{2}\bigg)\Delta t + \sigma \sqrt{\Delta t} Z} \bigg)
\end{align}

Where \(Z \sim N(0,1)\).

To simulate one GBM path efficiently, I am using vectorization and two
lines of \texttt{R} code. The in-built \texttt{R} function
\texttt{cumprod()} cumulatively multiplies the incremental factors in
the GBM evolution.

\vspace{2mm}

\setstretch{1.0}
\small

\begin{Shaded}
\begin{Highlighting}[]
\NormalTok{gbm_path <-}\StringTok{ }\ControlFlowTok{function}\NormalTok{(s0, mu, sigma, T, }\DataTypeTok{dt =} \DecValTok{1}\OperatorTok{/}\NormalTok{(T}\DecValTok{-1}\NormalTok{)) \{}
  
\CommentTok{# draw from the standard-normal distribution  }
\NormalTok{epsilons <-}\StringTok{ }\KeywordTok{rnorm}\NormalTok{(T}\DecValTok{-1}\NormalTok{) }
  
\CommentTok{# compute GBM path from t=0 to t=T}
\KeywordTok{cumprod}\NormalTok{(}\KeywordTok{c}\NormalTok{(s0,}\KeywordTok{exp}\NormalTok{((mu}\OperatorTok{-}\NormalTok{sigma}\OperatorTok{*}\NormalTok{sigma}\OperatorTok{/}\DecValTok{2}\NormalTok{)}\OperatorTok{*}\NormalTok{dt }\OperatorTok{+}\StringTok{ }\NormalTok{sigma}\OperatorTok{*}\NormalTok{epsilons}\OperatorTok{*}\KeywordTok{sqrt}\NormalTok{(dt))))}
\NormalTok{\}}
\end{Highlighting}
\end{Shaded}

\setstretch{1.25}
\normalsize

Below, I am using the \texttt{gbm\_path()} function to simulate 10 GBM
paths sharing the following characteristics:

\setstretch{1.1}
\small

\begin{tabular}{lll}

$S_0$ & $= 100$ & Stock price in $t=0$ is equal to \$100\\
$T$ & $= 252$ & Simulate 252 trading days, i.e. 1 calendar year from $S_0$ to $S_T$\\
$\mu$ & $= 0.07$ & Stock $S$ has an expected, continously compounded, annual return of 7\%\\
$\sigma$ & $= 0.3$ & Stock $S$ has constant, annual volatility of 30\%\\
$\Delta t$ & $= 1/251$ & If $t=0$ to $t=T$ represents one year, then $\Delta t = 1/(T-1)$\\

\end{tabular}

\normalsize
\setstretch{1.25}
\vspace{6mm}

\begin{figure}[H]

{\centering \includegraphics[width=0.76\linewidth]{vanilla-and-exotic-option-pricing_files/figure-latex/plot-gbm-1} 

}

\caption{10 Simulated GBM Paths with Equal Characteristics}\label{fig:plot-gbm}
\end{figure}

\newpage

\hypertarget{code-cross-check}{%
\subsubsection{Code Cross-Check}\label{code-cross-check}}

When simulating a GBM path with \(S_0 = 100\), \(\mu = 0.07\), and
\(T-t_0\) resembling one calendar year, the expected stock price at
expiry, \(E[S_T]\) is equal to:

\[E[S_T] = S_0\exp\big({\mu}\big) = 100\exp\big({0.07}\big) = 107.2508\]

To test the correctness of the \texttt{gbm\_path()} function, below I am
simulating \(n = 100,000\) GBM paths to estimate \(S_T\).

\singlespacing
\small

\begin{Shaded}
\begin{Highlighting}[]
\CommentTok{# for reproducibility}
\KeywordTok{set.seed}\NormalTok{(}\DecValTok{1}\NormalTok{)}

\CommentTok{# simulate 100k GBMs with low sigma}
\NormalTok{gbms <-}\StringTok{ }\KeywordTok{replicate}\NormalTok{(}\DecValTok{100000}\NormalTok{,}
    \KeywordTok{gbm_path}\NormalTok{(}\DataTypeTok{s0 =} \DecValTok{100}\NormalTok{, }\DataTypeTok{mu =} \FloatTok{0.07}\NormalTok{, }\DataTypeTok{sigma =} \FloatTok{0.0005}\NormalTok{, }\DataTypeTok{T =} \DecValTok{252}\NormalTok{))}

\CommentTok{# compute average stock price at expiry and 95% CIs}
\NormalTok{ST <-}\StringTok{ }\KeywordTok{mean}\NormalTok{(}\KeywordTok{tail}\NormalTok{(gbms,}\DecValTok{1}\NormalTok{))}
\NormalTok{sd <-}\StringTok{ }\KeywordTok{sd}\NormalTok{(}\KeywordTok{tail}\NormalTok{(gbms,}\DecValTok{1}\NormalTok{))}
\NormalTok{ci_upper <-}\StringTok{ }\NormalTok{ST }\OperatorTok{+}\StringTok{ }\NormalTok{(sd}\OperatorTok{/}\KeywordTok{sqrt}\NormalTok{(}\DecValTok{100000}\NormalTok{))}\OperatorTok{*}\KeywordTok{pnorm}\NormalTok{(}\FloatTok{0.975}\NormalTok{)}
\NormalTok{ci_lower <-}\StringTok{ }\NormalTok{ST }\OperatorTok{-}\StringTok{ }\NormalTok{(sd}\OperatorTok{/}\KeywordTok{sqrt}\NormalTok{(}\DecValTok{100000}\NormalTok{))}\OperatorTok{*}\KeywordTok{pnorm}\NormalTok{(}\FloatTok{0.975}\NormalTok{)}
\end{Highlighting}
\end{Shaded}

\setstretch{1.25}
\normalsize
\vspace{4mm}

\begin{table}[!h]

\caption{\label{tab:table}Simulation Results for Stock Price at Expiry}
\centering
\begin{tabular}[t]{cc}
\toprule
$\widehat{S_T}$ & 95\% CI\\
\midrule
107.2508 & [107.25099, 107.25071]\\
\bottomrule
\multicolumn{2}{l}{\rule{0pt}{1em}$n = 0.1\text{m}$, $S_0 = \$100$, $\mu=0.07$, $\sigma=0.0005$.}\\
\end{tabular}
\end{table}

The simulation confirms that the \texttt{gbm\_path()} function yields an
unbiased estimate of \(S_T\).

\newpage

\hypertarget{option-pricing-using-monte-carlo-simulation}{%
\subsection{Option Pricing using Monte-Carlo
Simulation}\label{option-pricing-using-monte-carlo-simulation}}

Under the Black-Scholes model, the investor is invariant to risk. Hence,
risky assets are discounted at the risk-free rate, \(r\). Likewise,
risky assets are expected to grow at the risk-free rate, \(r\). Hence,
when simulating GBM paths, \(\mu\) has no relevance and is set to
\texttt{mu\ =\ r} in the \texttt{gbm\_path()} function.

Obtaining the price for a European call option is straightforward.
First, let's simulate several GBM paths. Secondly, compute the pay-off
at expiry. Thirdly, discount all pay-offs to time \(t=0\). The average
of all present values of the simulated pay-offs is equal in expectation
to the option price obtained via the Black-Scholes formula:

\begin{align}
\text{V}^{\text{C}_{\text{E}}}_t = S_t F_N \big(d_1\big) - X \exp^{-r(T-t)} F_N \big(d_2\big) \label{bscall}
\end{align}

\setstretch{1.1}
\small

Where:

\(\begin{aligned} d_1 &= \frac{1}{\sigma\sqrt{T-t}}\bigg[\ln\bigg(\frac{S_t}{X}\bigg)+\bigg(r + \frac{\sigma^2}{2}\bigg)(T-t) \bigg] \notag\\ \vspace{2mm} d_2 &= d_1 - \sigma\sqrt{T-t} \notag\\ \vspace{2mm} F_N() &: \text{Cumulative distribution function of the standard normal distribution} \notag\\ \sigma &: \text{Annualized volatility of the share price } S_t \notag\\ T-t &: \text{Time to expiry in years} \notag\\ S_t &: \text{Share price at time } t \notag\\ X &: \text{Strike price of the option} \notag\\ r &: \text{Annualized, continuously compunded risk-free rate} \notag \end{aligned}\)

\setstretch{1.25}
\normalsize
\vspace{2mm}

Note that at the option's inception, \(t=0\), and for expiry in one
year, \(T=1\), the terms simplify to \(T-t = 1\) and \(S_t = S_0\).

\hypertarget{dissecting-the-black-scholes-equations-terms}{%
\subsubsection{Dissecting the Black-Scholes Equation's
Terms}\label{dissecting-the-black-scholes-equations-terms}}

In the following, I am developing a method to obtain estimates for a
vanilla option's delta, i.e.~the first derivative of the option price
with respect to the share price. Delta is of crucial importance as it is
used to hedge the option price against changes in the underlying asset.

To develop an approach to simulate an option's delta, it is useful to
first interpret the terms in Black-Scholes equation \eqref{bscall}. In
line with Nielsen (1992)\footnote{Lars Tyge Nielsen,
  \textit{"Understanding N(d1) and N(d2) : risk adjusted probabilities in the black-scholes model"},
  INSEAD, 1992.}, the terms resemble the following:

\begin{align}
\underbrace{\text{V}^{\text{C}}_t}_{\substack{\text{European Call}\\  \text{Price at time }t}} = \underbrace{S_t}_{\substack{\text{Present Value}\\  \text{of }\mathbb E[S_T]}}\underbrace{F_N \big(d_1\big)}_{\frac{\partial V^C_{t}}{\partial S_t}} - \underbrace{X \exp^{-r(T-t)}}_{\substack{\text{Present Value}\\  \text{of Strike Price}}} \underbrace{F_N \big(d_2\big)}_{\mathbb{P}(S_T>X)} \label{bscall2}
\end{align}

\textbf{\emph{Note:}}

\begin{enumerate}
\def\labelenumi{\arabic{enumi}.}
\tightlist
\item
  As established in the previous section, due to risk-neutrality,
  \(S_t = \mathbb E[S_T] \exp^{-r(T-t)}\), and likewise,
  \(\mathbb E[S_T] = S_t \exp^{r(T-t)}\). \vspace{1mm}
\item
  It is a common misconception to interpret the option's delta as the
  probability of the option ending up in the money at expiry. When it
  comes to a European call option, \(\text{Delta}^{\text{C}}=F_N(d_1)\)
  \emph{overestimates} the true probability,
  \(\mathbb P(S_T>X) = F_N(d_1 - \sigma\sqrt{T-t})\). This is due to the
  missing volatility term, \(\sigma\sqrt{T-t}\). Given delta, this
  probability can be calculated using the standard normal quantile
  function, \(Q_N(p)\):
  \[\mathbb P(S_T>X) = F_N \bigg(Q_N\big(\text{Delta}^{\text{C}}\big) - \sigma\sqrt{T-t} \bigg)\]
  In \texttt{R}, this can be computed as
  \texttt{pnorm(qnorm(d)\ -sigma*sqrt(T-t))}, where
  \(d=\text{Delta}^{\text{C}}\), \(\text{sigma}=\sigma\) (e.g.~implied
  volatility), and \(\text{T-t}\) is the time to maturity in years.
  \vspace{1mm}
\end{enumerate}

In the following section, I am using \eqref{bscall2} to simulate a
European call option's delta.

\newpage

\hypertarget{simulating-a-vanilla-options-delta}{%
\subsubsection{Simulating a Vanilla Option's
Delta}\label{simulating-a-vanilla-options-delta}}

It is straightforward for most options to determine the terms in
equation \eqref{bscall2} using Monte-Carlo simulation. One possible
exception, however, is the option's delta, which is given by
\(F_N(d_1)\) for a European call. More specifically,

\begin{enumerate}
\def\labelenumi{\arabic{enumi}.}
\tightlist
\item
  \(S_t\), \(X\), \(r\), and \(T-t\) are all given.
\item
  \(\mathbb{P}(S_T>X)\) can be estimated by computing the proportion of
  several GBM paths exceeding the strike price, \(X\) at expiry, \(T\).
\item
  \(\text{V}^{\text{C}}_t\), the option price at time \(t\), can be
  estimated by discounting and averaging the simulated pay-offs at
  expiry \(T\).
\end{enumerate}

Hence, using \eqref{bscall2} and Monte-Carlo simulation, the option's
delta for a European call option can be estimated by:

\begin{align}
\text{Delta}^{\text{C}} = \frac{\partial{V^C_t}}{{\partial S_t}} = \frac{1}{S_t} \bigg[ \text{V}^{\text{C}}_t + X \exp^{-r(T-t)} \mathbb{P}(S_T > X) \bigg] \label{deltaeq}
\end{align}

Note that \(S_t\), \(X\), \(r\), \(T\), and \(t\) are all exogenous and
the two variables obtainable by simulation are \(\mathbb{P}(S_T > X)\)
and \(\text{V}^{\text{C}}_t\). Hence, equation \eqref{deltaeq} reduces
to a scaled sum of two variables of the form \(aX + bY\). Thus, its
variance is equal to:

\begin{align}
\text{Var}\big[\hat{\text{Delta}}^\text{C}\big] = \frac{1}{S_t^2} \bigg[ \text{Var}\big[\hat{\text{V}}^{\text{C}}_t\big] + b^2\, \text{Var}\big[\hat{\mathbb P}(S_T > X)\big] + 2\, b\, \text{Cov}\big[\hat{\text{V}}^{\text{C}}_t,\, \hat{\mathbb P}(S_T > X)  \big] \bigg] \label{vardelta}
\end{align}

Where:

\(\begin{aligned} b &= X\exp^{-r(T-t)} \notag \end{aligned}\)

It follows the standard error for \(\hat{\text{Delta}}^\text{C}\):

\begin{align}
\text{SE}\big[\hat{\text{Delta}}^\text{C}\big] = \frac{\sqrt{\text{Var}\big[\hat{\text{Delta}}^\text{C}\big]}}{\sqrt{n}} \label{sedelta}
\end{align}

\hypertarget{application-simulating-the-price-and-delta-of-a-european-call}{%
\subsubsection{Application: Simulating the Price and Delta of a European
Call}\label{application-simulating-the-price-and-delta-of-a-european-call}}

In the following, I am comparing the price, the delta, and the
probability of exercise of a European call option obtained by the
Black-Scholes equation to the estimates obtained by the simulation
method as described in
\protect\hyperlink{simulating-a-vanilla-options-delta}{the previous
section}.

Consider a European call option with the following characteristics:

\setstretch{1.1}
\small

\begin{tabular}{lll}

$S_0$ & $= \$100$ & Stock price at $t=0$ is \$100\\
$\sigma$ & $= 0.2$ & The underlying asset has 20\% volatility\\
$X$ & $= \$100$ & The option's strike price is \$100\\
$r$ & $= 0.02$ & Risk-free rate is 2\%\\
$T-t$ & $= 1$ & 1 year to expiry\\

\end{tabular}

\setstretch{1.25}
\normalsize
\vspace{2mm}

The table below compares the results obtained by Monte-Carlo simulation
and the Black-Scholes equation.

\begin{table}[!h]

\caption{\label{tab:show-results}Black-Scholes Solution vs Monte-Carlo Estimates}
\centering
\begin{tabular}[t]{lrrc}
\toprule
\multicolumn{2}{c}{ } & \multicolumn{2}{c}{\makecell[c]{Monte-Carlo\\Simulation}} \\
\cmidrule(l{3pt}r{3pt}){3-4}
Variable & \makecell[c]{Black-Scholes\\Solution} & Estimate & 95\% CI\\
\midrule
$V^{C}_0$ & 8.916 & 8.913 & [8.9034, 8.9222]\\
$\mathbb{P}(S_T > X)$ & 0.500 & 0.500 & [0.4997, 0.5003]\\
$\text{Delta}^{\text{C}}$ & 0.579 & 0.579 & [0.5788, 0.5796]\\
\bottomrule
\multicolumn{4}{l}{\rule{0pt}{1em}$n=1.5\text{m}$, $S_0 = \$100$, $r=0.02$, $\sigma=0.2$, $X = \$100$.}\\
\end{tabular}
\end{table}

As seen in the table above, all variables' confidence intervals cover
the Black-Scholes solutions. More specifically, the simulated results
are accurate to at least two digits. The 95\% confidence interval for
the option price, \(\text{V}_{0}^{\text{C}}\), has a total width of just
\$0.02.

Also note the evidence for the claim that,
\(\mathbb{P}(S_T > X) = F_N(d_2)\), where \(\mathbb{P}(S_T > X)\) is
simulated by computing the proportion of GBMs ending up in the money
\(S_T > X = \$100\).

\hypertarget{asian-option-pricing}{%
\section{Asian Option Pricing}\label{asian-option-pricing}}

The key difference between Asian and vanilla options is how the pay-off
depends on the underlying's price. More specifically, the pay-off of an
Asian option is dependent on the \emph{average} stock price,
\(\overline{S_T}\) at finite, different time intervals. Assuming only
one time interval, the pay-offs for Asian call and put options at expiry
are as follows:

\textbf{\emph{Average Rate Asian Call Option Pay-Off}}

\[C_T = \text{max} \bigg( \frac{1}{T} \sum_{t=0}^{T} S_t - X,0 \bigg)\]

\textbf{\emph{Average Rate Asian Put Option Pay-Off}}

\[P_T = \text{max} \bigg(X - \frac{1}{T} \sum_{t=0}^{T} S_t,0 \bigg)\]

Hence, the pay-off of an Asian option is dependent on the share price
over the \emph{lifetime} of the option, whereas for vanilla options, the
pay-off is only dependent on the (one) share price at expiry.

To illustrate this, let's simulate 5 GBMs, resembling 5 different assets
over time. Below, the first plot resembles the actual price development
of the underlying assets. The second plot, however, depict the
\emph{cumulative artihmetic averages} of the underlying share prices.
This is the price function the Asian option is dependent on.

\vspace{3mm}

\begin{figure}[H]

{\centering \includegraphics[width=0.84\linewidth]{vanilla-and-exotic-option-pricing_files/figure-latex/asian-option-dependencies-1} 

}

\caption{Price Evolution of Underlying Assets Versus Asian Options' Dependencies}\label{fig:asian-option-dependencies}
\end{figure}

As seen in figure 2, the Asian option's stock price dependencies are far
less volatile as compared to vanilla options, which are dependent on the
actual share price. Hence there should also be far less variation in the
Asian option's delta when compared to vanilla options, making
delta-hedging generally more accurate.

\hypertarget{simulating-an-asian-call-options-price-and-probability-of-exercise}{%
\subsection{Simulating an Asian Call Option's Price and Probability of
Exercise}\label{simulating-an-asian-call-options-price-and-probability-of-exercise}}

Assuming an \emph{average rate}, one-period Asian call option, there are
several similarities to a European call option. First, it spawns across
one specific time period. Secondly, it can only be exercised at expiry.
Thirdly, the pay-off is the difference between a function of the
underlying asset's price and the fixed strike price \(X\), or zero. The
\emph{only} difference is that the European call option depends on
\(S_t\) and \(\mathbb E[S_T]\), whereas the Asian call option depends on
\(\overline{S}_t = \frac{1}{t} \sum^{t}_{i = 0} S_i\) and
\(\mathbb E\big[\overline{S}_T\big] = \mathbb E\big[ \frac{1}{T} \sum^{T}_{i = 0} S_i \big]\).
Hence, at any point in time, the Asian option's price is at least
partially dependent on the \emph{past} prices of the underlying asset.
Thus, it is sensible to distinguish between pricing an Asian option at
inception, and pricing an Asian option after its inception, where
\(t>0\).

\vspace{2mm}

\textbf{\emph{Simulating the Price and Probability of Exercise at
Inception of the Option}}

Simulating an Asian option's price, and the probability of exercise,
\(\mathbb P(S_T>X)\), is similar to the approach of a vanilla option,
with one additional step (i.e.~the average rate transformation of the
simulated GBM paths). The proposed method is as follows:

\begin{enumerate}
\def\labelenumi{\arabic{enumi}.}
\tightlist
\item
  Simulate several GBM paths. Result: Many estimates for \(S_T\).
\item
  Compute the \emph{arithmetic mean} of the simulated GBM paths. Result:
  Many estimates for \(\overline{S}_T\).
\item
  From (2), compute the Asian call option's pay-off,
  i.e.~\(V^C_T = \text{max} \big(\overline{S}_T - X,0 \big)\). Result:
  Many estimated pay-offs at expiry, \(T\).
\item
  From (2), compute the proportion of \(\overline{S}_T > X\) via
  \texttt{mean(ST\ \textgreater{}\ X)}. This yields the estimate for
  \(\mathbb P\big(\overline{S}_T > X \big)\).
\item
  Discount the pay-offs in (3) from \(T\) to \(t=0\). Result: Many
  present values of the estimated future pay-offs.
\item
  The average of the present value of the future pay-offs estimates the
  Asian option price at time \(t=0\).
\end{enumerate}

\vspace{2mm}

\textbf{\emph{Simulating an Asian Option's Price and Probability of
Exercise After Inception}}

At any point in time \(t\), where \(T>t>0\), the expected value of the
Asian option's price dependency, \(\overline{S}_T\), is a weighted sum
of the following two elements:

\begin{align}
\mathbb E\big[\overline{S}_T\big] &= w_1 \; \overline{S}_t \: + \: w_2 \; \mathbb E\big[ \overline{S}_T \big] \notag\\
\vspace{2mm}
\mathbb E\big[\overline{S}_T\big] &= w_1 \; \frac{1}{t+1} \sum^{t}_{t=0} S_i \: + \: w_2 \; \mathbb E\bigg[ \frac{1}{T-t}\sum^{T}_{i=t+1} S_i\bigg]
\end{align}

\setstretch{1.1}
\small

Where:

\(\begin{aligned} w_1 &= t/T \notag\\ \vspace{1mm} w_2 &= (T-t)/T \notag\\ \vspace{1mm} S_i &= \text{Price of risky asset at time } t=i. \notag \label{weighted_sum} \end{aligned}\)

\setstretch{1.25}
\normalsize

\vspace{2mm}

\textbf{Algorithm:}

\begin{enumerate}
\def\labelenumi{\arabic{enumi}.}
\tightlist
\item
  Compute the arithmetic average price of the underlying asset from
  \(S_0\) to \(S_t\), where \(t=0\) is the date of the inception of the
  option, i.e.~\(\overline{S}_{0,t}= \frac{1}{t} \sum^{t}_{i = 0} S_i\).
\item
  Simulate many GBM paths from \(S_t\) to \(S_T\). Result: Many
  potential price paths from \(t\) to \(T\).
\item
  For each price evolution in (2), compute the arithmetic average from
  \(t\) to \(T\). Result: Many estimates for \(\overline{S}_{t,T}\).
\item
  Multiply each \(\overline {S}_{t,T}\) by \(\frac{T-t}{T}\) and add
  \(\frac{t}{T} \overline S_{0,t}\). Result: Many estimates for
  \(\overline{S}_T\).
\item
  Compute the Asian call option's pay-off,
  i.e.~\(V^C_T = \text{max} \big({S}_T - X,0 \big)\). Result: Many
  estimates for pay-offs at time \(T\).
\item
  For (5), compute the proportion of \({S}_T > X\) via
  \texttt{mean(ST\_hats\ \textgreater{}\ X)}. This yields the estimate
  for \(\mathbb P\big({S}_T > X \big)\).
\item
  For (5), discount the estimated pay-offs from time \(T\) to \(t\).
  Result: Many present values of the future estimated pay-offs.
\item
  The arithmetic average of (7) estimates the option price at time
  \(t\).
\item
  Applying the Central Limit Theorem, it is straightforward to estimate
  a confidence interval for (8) by using the standard error
  \(\text{SE}=\frac{s}{\sqrt{n}}\), where \(n\) is the number of
  simulated pay-offs in (7) and \(s\) is the sample standard deviation
  of (7).
\end{enumerate}

For computational efficiency, instead of simulating GBM paths for each
point in time, {[}\(t=0\), \(t=1\), \ldots, \(t=T\){]}, I am only once
simulating 1,000,000 GBMs. To obtain estimates for {[}\(t=1\), \ldots,
\(t=T\){]}, I am consecutively subtracting one day at a time from the
end of these paths, and scaling them by \(S_t/S_0\).

\hypertarget{application-asian-options-price-probability-of-exercise-and-empirical-delta-for-a-risky-asset}{%
\section{Application: Asian Option's Price, Probability of Exercise, and
Empirical Delta for a Risky
Asset}\label{application-asian-options-price-probability-of-exercise-and-empirical-delta-for-a-risky-asset}}

Consider one risky asset having the following characteristics:

\setstretch{1.1}
\small

\begin{tabular}{lll}

$S_0$ & $= \$100$ & Stock price at $t=0$ is \$100;\\
$\sigma$ & $= 0.25$ & The underlying asset has 25\% volatility.\\

\end{tabular}

\setstretch{1.25}
\normalsize
\vspace{2mm}

Assume that the risky asset has the following price evolution over 1
calendar year (252 trading days). Further assume that at any point in
time \(t\), the future price of the risky asset, \(S_{t+i}\) where
\(i>0\), is unknown.

\vspace{2mm}

\begin{figure}[H]

{\centering \includegraphics[width=0.8\linewidth]{vanilla-and-exotic-option-pricing_files/figure-latex/simulate-one-risky-asset-1} 

}

\caption{Price Evolution of Hypothetical Risky Asset}\label{fig:simulate-one-risky-asset}
\end{figure}

Furthermore, consider one European call option and one Asian call
option, both having the same underlying as described above, sharing the
following characteristics:

\setstretch{1.1}
\small

\begin{tabular}{lll}

$X$ & $= \$100$ & The option's strike price is \$100\\
$r$ & $= 0.02$ & Risk-free rate is 2\%\\
$T-t$ & $= 1$ & 1 year to expiry\\

\end{tabular}
\setstretch{1.25}
\normalsize
\vspace{2mm}

For the risky asset in figure 3, I am using the methods as described in
the previous sections to simulate daily prices and probabilities of
exercise for both a European and an average rate Asian call option.

\begin{figure}
\centering
\includegraphics{vanilla-and-exotic-option-pricing_files/figure-latex/plot-results-1.pdf}
\caption{European Versus Asian Call Price and Probability of Exercise}
\end{figure}

\hypertarget{empirical-delta}{%
\subsection{Empirical Delta}\label{empirical-delta}}

Mathematically, an option's delta is the first derivative of the option
price with respect to the price of the underlying asset. The
discrete-time equivalent is the change in the option price divided by
the change in the underlying:

\begin{align}
\text{Delta}_t &= \frac{\partial V_{t}}{\partial S_t} \notag\\
\vspace{20mm}
&\approx \frac{\Delta V_{t}}{\Delta S_t} = \frac{V_t - V_{t-1}}{S_t-S_{t-1}}
\label{empiricaldelta}
\end{align}

As opposed to the continuous-time derivative, the discrete-time delta in
equation \eqref{empiricaldelta} is not bound by \([0,1]\). Empirically,
this delta represents the quantity fraction necessary to perfectly hedge
a portfolio consisting of an option and its underlying asset. Thus, the
empirical delta is found by setting the \emph{change} in this portfolio,
\(\Delta \text{PF}\), to zero:

\begin{align}
\Delta \text{PF}_t &= \Delta V_t + \text{Delta}_t \: \Delta S_t \notag\\
0 &= \Delta V_t + \text{Delta}_t \: \Delta S_t \notag\\
\vspace{4mm}
\text{Delta}_t &= - \frac{\Delta V_{t}}{\Delta S_t} \notag
\end{align}

Thus, the negative value of the discrete-time delta in equation
\eqref{empiricaldelta} is the empirical quantity of the underlying asset
needed to perfectly hedge the option portfolio.

Below, I am comparing the empirical deltas between European options and
Asian options for the same underlying asset as in figures 3 and 4.

\begin{figure}
\centering
\includegraphics{vanilla-and-exotic-option-pricing_files/figure-latex/plot-empirical-delta-1.pdf}
\caption{European Versus Asian Call Price and Empirical Delta}
\end{figure}

Noteworthy, the average rate option's delta goes to zero as the option
approaches expiry. As seen in equation \eqref{weighted_sum}, this is
because the option's weight on past prices of the risky asset increases
to one as time goes by. This mechanism also explains the behavior of the
probability of exercise in figure 4. At a certain point in time, the
Asian option's pay-off \enquote{sticks} to a specific value, and it
becomes even more sticky as time goes by.

\hypertarget{conclusion}{%
\section{Conclusion}\label{conclusion}}

In this paper, I propose methods to estimate vanilla and exotic options'
prices and probability of exercise using Monte Carlo simulation in
\texttt{R}. The computational efficiency of these methods relies on;

\begin{enumerate}
\def\labelenumi{\arabic{enumi}.}
\tightlist
\item
  Usage of vectorization, and
\item
  Recycling of the simulated Geometric Brownian Motion paths.
\end{enumerate}

\protect\hyperlink{application-asian-options-price-probability-of-exercise-and-empirical-delta-for-a-risky-asset}{Section
3} illustrates the differences between a European and an average rate
Asian option for the same underlying asset. Due to the function applied
on the price evolution of the underlying asset, the average rate Asian
option generally exposes less variability in its price, probability of
exercise, and delta. More specifically, the average rate Asian option
has decreasing variability in all investigated dimensions as time goes
by. When it comes to the Asian option's delta, it inevitably approaches
zero over time, as the underlying's future prices have continuously less
impact on the option's pay-off.

In general, the cumulative average transformation of the risky asset's
price in the Asian option's pay-off greatly reduces uncertainty. This
reduced risk is revealed in lower variability of the option price,
delta, and probability of exercise. Naturally, this price transformation
also the potential return, explaining the lower option price at
inception.

\newpage

\hypertarget{references}{%
\section*{References}\label{references}}
\addcontentsline{toc}{section}{References}

\hypertarget{refs}{}
\leavevmode\hypertarget{ref-zoo}{}%
Achim Zeileis. 2021. \emph{Package ``Zoo''}.
\url{https://cran.r-project.org/web/packages/zoo/}.

\leavevmode\hypertarget{ref-ggthemes}{}%
Arnold, J. B. et al. 2019. \emph{Package ``Ggthemes''}.
\url{https://cran.r-project.org/web/packages/ggthemes/ggthemes.pdf}.

\leavevmode\hypertarget{ref-gridExtra}{}%
Auguie, B. 2017. \emph{Package ``gridExtra''}.
\url{https://cran.r-project.org/web/packages/gridExtra/gridExtra.pdf}.

\leavevmode\hypertarget{ref-dt}{}%
Dowle, M. et al. 2020. \emph{Package ``Data.table''}.
\url{https://cran.r-project.org/web/packages/data.table/data.table.pdf}.

\leavevmode\hypertarget{ref-stringr}{}%
Hadley Wickham. 2019. \emph{Package ``Stringr''}.
\url{https://cran.r-project.org/web/packages/stringr/}.

\leavevmode\hypertarget{ref-scales}{}%
Hadley Wickham, Dana Seidel, RStudio. 2020. \emph{Package ``Scales''}.
\url{https://cran.r-project.org/web/packages/scales/}.

\leavevmode\hypertarget{ref-kableExtra}{}%
Hao Zhu. 2021. \emph{Package ``kableExtra''}.
\url{https://cran.r-project.org/web/packages/kableExtra/}.

\leavevmode\hypertarget{ref-matrixStats}{}%
Henrik Bengtsson. 2021. \emph{Package ``matrixStats''}.
\url{https://cran.r-project.org/web/packages/matrixStats/}.

\leavevmode\hypertarget{ref-iriz}{}%
Irizarry, Rafael A. 2020. \emph{Introduction to Data Science}.
\url{https://rafalab.github.io/dsbook/}.

\leavevmode\hypertarget{ref-caTools}{}%
Jarek Tuszynski. 2021. \emph{Package ``caTools''}.
\url{https://cran.r-project.org/web/packages/caTools/}.

\leavevmode\hypertarget{ref-xts}{}%
Jeffrey A. Ryan. 2020. \emph{Package ``Xts''}.
\url{https://cran.r-project.org/web/packages/xts/}.

\leavevmode\hypertarget{ref-RcppRoll}{}%
Kevin Ushey. 2018. \emph{Package ``Rcpproll''}.
\url{https://cran.r-project.org/web/packages/RcppRoll/}.

\leavevmode\hypertarget{ref-Nielsen}{}%
Lars Tyge Nielsen. 1992. \emph{Understanding N(d1) and N(d2) : Risk
Adjusted Probabilities in the Black-Scholes Model}. INSEAD.

\leavevmode\hypertarget{ref-R-base}{}%
R Core Team. 2020. \emph{R: A Language and Environment for Statistical
Computing}. Vienna, Austria: R Foundation for Statistical Computing.
\url{https://www.R-project.org}.

\leavevmode\hypertarget{ref-fasttime}{}%
Simon Urbanek. 2016. \emph{Package ``Fasttime''}.
\url{https://cran.r-project.org/web/packages/fasttime/}.

\leavevmode\hypertarget{ref-lubridate}{}%
Spinu, V. 2020. \emph{Package ``Lubridate''}.
\url{https://cran.r-project.org/web/packages/lubridate/lubridate.pdf}.

\leavevmode\hypertarget{ref-tidyverse}{}%
Wickham, H. 2019. \emph{Package ``Tidyverse''}.
\url{https://cran.r-project.org/web/packages/tidyverse/tidyverse.pdf}.

\leavevmode\hypertarget{ref-extrafont}{}%
Winston Chang. 2014. \emph{Package ``Extrafont''}.
\url{https://cran.r-project.org/web/packages/extrafont/}.

\leavevmode\hypertarget{ref-knitr}{}%
Xie, Y. et al. 2020. \emph{Package ``Knitr''}.
\url{https://cran.r-project.org/web/packages/knitr/knitr.pdf}.

\end{document}
